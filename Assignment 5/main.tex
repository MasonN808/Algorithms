\documentclass{article}
\usepackage[utf8]{inputenc}
% Margins
\usepackage[a4paper, total={6in, 10in}]{geometry}
\usepackage{listings}
\usepackage{float}
\lstset
{ %Formatting for code in appendix
    language=Matlab,
    basicstyle=\footnotesize,
    numbers=left,
    stepnumber=1,
    showstringspaces=false,
    tabsize=1,
    breaklines=true,
    breakatwhitespace=false,
}

\title{Assignment 5 - Code Evaluation}
\author{Mason Nakamura}
\date{December 2021}





\usepackage{tabularx}
% For tabularx
\newcolumntype{Y}{>{\centering\arraybackslash}X}
\begin{document}

\maketitle

\section{LinkedObjects Class}
    \begin{lstlisting}
import java.lang.reflect.Array;
import java.util.ArrayList;

public class LinkedObjects {
    ArrayList<Integer> num_vertices = new ArrayList<>();
    public void linked_objects(String[] lines) {
        // use these indices throughout for loops
        int index_start;
        int index_end = 0;
        // for printing purposes
        int graph_id = 0;
        // for vertex starting index
        boolean indexIs0 = false;
        // Read the lines up to add edge case
        while (index_end <= lines.length) {
            graph_id += 1;
            index_start = index_end;
            int adj_list_length = 0;
            // get indices for for loop for edges
            for (int i = index_start; i < lines.length; i++) {
                String line = lines[i];
                index_start += 1;
                // case for line is blank;
                if (line.isBlank()) {
                    // skip the line
                    continue;
                }
                // case for line is a comment
                if (line.charAt(1) == '-' & line.charAt(0) == '-') {
                    // skip the line
                    continue;
                }
                // make an array of the words from the line
                String[] words = line.split(" ");
                // case for declaring new graph
                if (words[0].equals("new")) {
                    // print graph_id
                    System.out.println("Graph Number: " + graph_id);
                    //skip the line
                    continue;
                }
                // case for adding vertex
                if (words[0].equals("add") & words[1].equals("vertex")) {
                    adj_list_length += 1;
                    if (words[2].equals("0")) {
                        indexIs0 = true;
                    }
                    continue;
                }
                //check if hit edge case
                if (words[0].equals("add") & words[1].equals("edge")) {
                    // subtract an index since we previously added it but didn't need to since we entered the edge case
                    index_start -= 1;
                    break;
                }

            }

            // create instance of array given number of vertices declared
            index_end = index_start + 1;
            //make array of vertex objects
            Vertex[] array = new Vertex[adj_list_length];
            // check starting index

            // create first vertex objects in array given index
            if (indexIs0) {
                for (int i = 0; i < adj_list_length; i++) {
                    // from Vertex class
                    Vertex vertex = new Vertex();
                    vertex.connecting_vertex = i;
                    vertex.label = i;
                    vertex.next = null;
                    vertex.neighbors = new ArrayList<Vertex>();
                    array[i] = vertex;
                }

            } else {// shift index by 1
                for (int i = 1; i < adj_list_length + 1; i++) {
                    Vertex vertex = new Vertex();
                    vertex.connecting_vertex = i;
                    vertex.label = i;
                    vertex.next = null;
                    vertex.neighbors = new ArrayList<Vertex>();
                    array[i - 1] = vertex;
                }
            }

            // use index_start to continue in the for loop
            for (int i = index_start; i < index_end; i++) {
                // case if reached to end of file/(array of lines)
                if (lines.length < index_end) {
                    break;
                }
                String line = lines[i];
                String[] words = line.split(" ");
                // check if line is blank; if so, move to next graph
                if (line.isBlank()) {
                    break;
                }

                // check if there are two spaces between the weight and the connecting vertex
                // if so, move up an index
                if (words[5].equals("")){
                    words[5] = words[6];
                }

                // case for adding edge
                if (words[0].equals("add") & words[1].equals("edge")) {
                    index_end += 1;
                    // you need to define vertex1 and vertex2 since not doing so creates a pointer infinitely pointing to itself
                    if (indexIs0) {// if index starts at 0 don't subtract 1
                        Vertex vertex1 = new Vertex();
                        vertex1.origin_vertex = Integer.parseInt(words[2]);
                        vertex1.connecting_vertex = Integer.parseInt(words[4]);
                        vertex1.weight = Integer.parseInt(words[5]);
                        vertex1.label = vertex1.connecting_vertex;
                        Vertex head1 = array[vertex1.origin_vertex - 1];
                        while (head1.next != null) {
                            head1 = head1.next;
                        }
                        head1.next = vertex1;

                    } else {// subtract 1 to keep indices same
                        Vertex vertex1 = new Vertex();
                        vertex1.origin_vertex = Integer.parseInt(words[2]);
                        vertex1.connecting_vertex = Integer.parseInt(words[4]);
                        vertex1.weight = Integer.parseInt(words[5]);
                        vertex1.label = vertex1.connecting_vertex;
                        Vertex head1 = array[vertex1.origin_vertex - 1];
                        while (head1.next != null) {
                            head1 = head1.next;
                        }
                        head1.next = vertex1;
                    }
                }
            }


            Vertex[] copy_vertexes = array.clone();
            for(Vertex i: copy_vertexes){
                System.out.print("[" + i.label+ "]" + " ");
                while (i.next != null){
                    i = i.next;
                    System.out.print(i.connecting_vertex + "(" + i.weight + ")" + " ");
                }
                System.out.println();
            }

        }
    }

    // to insert row into 2-D array in
    public int[][] insertRow(int[][] m, int r, int[] data) {
        int[][] out = new int[m.length + 1][];
        for (int i = 0; i < r; i++) {
            out[i] = m[i];
        }
        out[r] = data;
        for (int i = r + 1; i < out.length; i++) {
            out[i] = m[i - 1];
        }
        return out;
    }

    public ArrayList<int[][]> matrices(String[] lines) {
        // use these indices throughout for loops
        int index_start;
        int index_end = 0;
        // for vertex starting index
        boolean indexIs0 = false;
        //initialize an arraylist of 2-arrays (matrices) so that we can pull the individual graphs in the testing file
        ArrayList<int[][]> matrices = new ArrayList<>();
        while (index_end <= lines.length) {
            index_start = index_end;
            // don't need rows and columns since adjacency matrix is symmetric ==> rows = columns
            int vertices = 0;
            // get indices for for loop for edges
            for (int i = index_start; i < lines.length; i++) {
                String line = lines[i];
                index_start += 1;
                // case for line is blank;
                if (line.isBlank()) {
                    // skip the line
                    continue;
                }
                // case for line is a comment
                if (line.charAt(1) == '-' & line.charAt(0) == '-') {
                    // skip the line
                    continue;
                }
                // make an array of the words from the line
                String[] words = line.split(" ");
                // case for declaring new graph
                if (words[0].equals("new")) {
                    //skip the line
                    continue;
                }
                // case for adding vertex
                if (words[0].equals("add") & words[1].equals("vertex")) {
                    vertices += 1;
                    if (words[2].equals("0")) {
                        indexIs0 = true;
                    }
                    continue;
                }
                //check if hit edge case
                if (words[0].equals("add") & words[1].equals("edge")) {
                    index_start -= 1;
                    break;
                }

            }
            // add number of vertices to list
            num_vertices.add(vertices);
            // create instance of matrix
            index_end = index_start + 1;
            int[][] matrix = new int[0][3];
            // use index_start to continue in the for loop
            for (int i = index_start; i < index_end; i++) {
                // case if reached to end of file/(array of lines)
                if (lines.length < index_end) {
                    break;
                }
                String line = lines[i];
                String[] words = line.split(" ");
                // check if line is blank; if so, move to next graph
                if (line.isBlank()) {
                    break;
                }

                // check if there are two spaces between the weight and the connecting vertex
                // if so, move up an index
                if (words[5].equals("")){
                    words[5] = words[6];
                }

                // case for adding edge
                if (words[0].equals("add") & words[1].equals("edge")) {
                    index_end += 1;
                    // use Integer.parseInt to convert string to int
                    if (indexIs0) {// if index starts at 0 don't subtract 1
                        // make array of data for added edge
                        int[] newData = new int[]{Integer.parseInt(words[2]) + 1, Integer.parseInt(words[4]) + 1, Integer.parseInt(words[5])};
                        // append the array to the matrix
                        matrix = insertRow(matrix, matrix.length, newData);

                    } else {// subtract 1 to keep indices same
                        // make array of data for added edge
                        int[] newData = new int[]{Integer.parseInt(words[2]), Integer.parseInt(words[4]), Integer.parseInt(words[5])};
                        // append the array to the matrix
                        matrix = insertRow(matrix, matrix.length, newData);
                    }
                }
            }

//            for (int i = 0; i < matrix.length; i++) {
//                for (int j = 0; j < matrix[i].length; j++) {
//                    System.out.print(matrix[i][j] + " ");
//                }
//                System.out.println();
//            }
//            System.out.println();
            matrices.add(matrix);
        }
        return matrices;
    }
}

\end{lstlisting}

In this class I used what I did in assignment 4, except I adapted it to directed graphs (fairly easy).  I have two methods. The first being the linked\_objects method which demonstrates that I constructed a linked object representation for each graph, outputting an adjacency list for visual purposes. The next methods called matrices() on line 165 was used as input for my SSSP algorithm where it gave the algorithm matrices to find the shortest path.

\section{SSSP Class}
    \begin{lstlisting}
import java.util.ArrayList;

public class SSSP {
    static void bellman_ford(int[][] graph, int V, int E, int src) {
        // Initialize distance of all vertices as very big value.
        int[] dis = new int[V];
        for (int i = 0; i < V; i++)
            dis[i] = Integer.MAX_VALUE;

        // initialize distance of source as 0
        dis[src-1] = 0;

        // RELAX
        for (int i = 0; i < V - 1; i++) {
            for (int j = 0; j < E; j++) {
                // subtract 1 if !indexis0
                if (dis[graph[j][0]-1] != Integer.MAX_VALUE && dis[graph[j][0]-1] + graph[j][2] < dis[graph[j][1]-1]) {
                    dis[graph[j][1]-1] = dis[graph[j][0]-1] + graph[j][2];
                }
            }
        }
        // check if there is an infinite cycle
        for (int i = 0; i < E; i++) {
            int x = graph[i][0];
            int y = graph[i][1];
            int weight = graph[i][2];
            if (dis[x-1] != Integer.MAX_VALUE && dis[x-1] + weight < dis[y-1]) {
                System.out.println("Graph contains negative weight cycle");
            }
        }

        System.out.println("Vertex \t\t Distance from Source");
        for (int i = 0; i < V; i++)
            System.out.println(i+1 + "\t\t\t " + dis[i]);
    }
}

    \end{lstlisting}
    The SSSP Class implements the BellmanFord Algorithm to find the shortest path on a directed weighted graph.  In this case, we used adjacency matrices to find the shortest paths.  We first initialize all the distances for each node in the graph as infinity (MAX\_VALUE) at line 7.  Then we apply a relax function on all the nodes in the graph and every edge in the graph checking for smallest distance relative to all other combinations of nodes and edges.  Then on line 23, we check if the algorithm gets stuck in an infinite cycle.  The complexity is $O(|V||E|)$ since we go through every edge on every vertex (the nested for loop in line 14).


\newpage
\section{Heist Class}
    \begin{lstlisting}
import java.lang.reflect.Array;
import java.util.ArrayList;
import java.util.Arrays;

public class Heist implements Cloneable{
    public static void spice_heist(String[] lines) {
        // use these indices throughout for loops
        final int NUMBER_OF_SPICES = 4;
        final int NUMBER_OF_KNAPSACKS = 5;
        int index_start = 0;
        int index_end = 0;
        int spice_index = 0;
        boolean onKnapsack = false;
        Spice[] spices = new Spice[NUMBER_OF_SPICES];
        // use unit_prices to sort the spices array
        float[] unit_prices = new float[NUMBER_OF_SPICES];
        ArrayList<Integer> knapsack_capacities = new ArrayList<>();
//        while (index_end <= lines.length) {
//            index_start = index_end;
        int last_index = 0;
        for (int i = index_start; i < lines.length; i++) {
            String line = lines[i];
            index_start += 1;
            // case for line is blank;
            if (line.isBlank()) {
                // skip the line
                continue;
            }
            // case for line is a comment
            if (line.charAt(0) == '-' & line.charAt(1) == '-') {
                // skip the line
                continue;
            }
            // make an array of the words from the line separated by ";"
            String[] words = line.split(";");
            // attributes for Spice object
            String[] spice_attributes = new String[3];
            for (int j = 0; j < words.length; j++) {
                words[j] = words[j].replaceAll("\\s", "");
                // separate the words in each element of words by "="
                String[] subwords = words[j].split("=");
                if (subwords[0].equals("knapsackcapacity")) {
                    onKnapsack = true;
                    last_index = index_start;
                    break;
                }
                if (!onKnapsack) {
                    // populate the spice attributes
                    spice_attributes[j] = subwords[1];
                }
            }
            if (!onKnapsack) {
                Spice new_spice = new Spice();
                new_spice.color = spice_attributes[0];
                new_spice.total_price = Float.parseFloat(spice_attributes[1]);
                new_spice.quantity = Integer.parseInt(spice_attributes[2]);
                //append the unit_price to array
                unit_prices[spice_index] = new_spice.total_price / new_spice.quantity;
                new_spice.unit_price = unit_prices[spice_index];
                //append the Spice object in an array
                spices[spice_index] = new_spice;
                spice_index += 1;
            }else{
                break;
            }
        }

        for (int i = last_index-1; i < lines.length; i++){
            String line = lines[i];
            String[] words = line.split(";");
            // attributes for Spice object
//            String[] spice_attributes = new String[3];
            for (int j = 0; j < words.length; j++) {
                words[j] = words[j].replaceAll("\\s", "");
                // separate the words in each element of words by "="
                String[] subwords = words[j].split("=");
                if (subwords[0].equals("knapsackcapacity")) {
                    knapsack_capacities.add(Integer.parseInt(subwords[1]));
                }
                else{
                    System.out.println("ERROR in Heist line 82");
                }
            }
        }
        RelativeInsertionSort_Decreasing RIS = new RelativeInsertionSort_Decreasing();
        // relatively sort unit_prices and spices
        RIS.relative_insertionSort_decreasing(unit_prices, spices);
        // start filling each knapsack and print results
        int out_index = 0;
        for (int i=0; i < knapsack_capacities.size(); i++){
            onKnapsack = false;
            index_start = 0;
            spice_index = 0;
            Spice[] spices2 = new Spice[NUMBER_OF_SPICES];
            // use unit_prices to sort the spices array
            unit_prices = new float[NUMBER_OF_SPICES];
            for (int k = index_start; k < lines.length; k++) {
                String line = lines[k];
                index_start += 1;
                // case for line is blank;
                if (line.isBlank()) {
                    // skip the line
                    continue;
                }
                // case for line is a comment
                if (line.charAt(0) == '-' & line.charAt(1) == '-') {
                    // skip the line
                    continue;
                }
                // make an array of the words from the line separated by ";"
                String[] words = line.split(";");
                // attributes for Spice object
                String[] spice_attributes = new String[3];
                for (int j = 0; j < words.length; j++) {
                    words[j] = words[j].replaceAll("\\s", "");
                    // separate the words in each element of words by "="
                    String[] subwords = words[j].split("=");
                    if (subwords[0].equals("knapsackcapacity")) {
                        onKnapsack = true;
                        last_index = index_start;
                        break;
                    }
                    if (!onKnapsack) {
                        // populate the spice attributes
                        spice_attributes[j] = subwords[1];
                    }
                }
                if (!onKnapsack) {
                    Spice new_spice = new Spice();
                    new_spice.color = spice_attributes[0];
                    new_spice.total_price = Float.parseFloat(spice_attributes[1]);
                    new_spice.quantity = Integer.parseInt(spice_attributes[2]);
                    //append the unit_price to array
                    unit_prices[spice_index] = new_spice.total_price / new_spice.quantity;
                    new_spice.unit_price = unit_prices[spice_index];
                    //append the Spice object in an array
                    spices2[spice_index] = new_spice;
                    spice_index += 1;
                }else{
                    break;
                }
            }


            Spice[] copy_spices = spices2.clone();

            RelativeInsertionSort_Decreasing RIS2 = new RelativeInsertionSort_Decreasing();
            // relatively sort unit_prices and spices
            RIS2.relative_insertionSort_decreasing(unit_prices, copy_spices);


            int sack_size = knapsack_capacities.get(i);
            // how much the knapsack is worth
            float worth = 0;
            // how many differing color spices were used
            int red = 0, green = 0, blue = 0, orange = 0;
            int index = 0;
            int j = 0;
            while (sack_size != 0 && copy_spices[copy_spices.length-1].quantity != 0){
                while (copy_spices[j].quantity != 0){
//                for (int j = 0; j < copy_spices.length; j++){
                    index += 1;
//                    if(copy_spices[j].quantity != 0){
                        sack_size -= 1;
                        worth += copy_spices[j].unit_price;
                        copy_spices[j].quantity -= 1;
//                        System.out.println(copy_spices[1].quantity);
                        switch (copy_spices[j].color) {
                            case "red" -> red += 1;
                            case "green" -> green += 1;
                            case "blue" -> blue += 1;
                            case "orange" -> orange += 1;
                        }
                        if (sack_size == 0){
                            break;
                        }
//                    }
//                    if(index == 3){
//                        break;
//                    }
                }
                j+=1;

            }
            int index1 = 0;
            int[] amounts = new int[]{red, green, blue, orange};
            String[] colors = new String[]{"red", "green", "blue", "orange"};
            String amount = "";
            for (int p=0; p< amounts.length; p++){
                int times = 0;
                while (amounts[p] != 0){
                    amounts[p] -= 1;
                    times += 1;
                }
                amount += " " + times + " scoops of " + colors[index1] + ",";
                index1 += 1;
            }

            System.out.println("Knapsack of capacity " + knapsack_capacities.get(i) + " is worth " + worth + " and " +
                    "contains " + amount);
            out_index += 1;
        }
    }


}

    \end{lstlisting}
    The Heist class has the spice\_heist() method used to solve the Knapsack problem.  We first parse the spice file from line 21 to line 84 where we populated several arrays to help in our outputs later on.  On line 85, we implemented a relative sorting algorithm to sort two arrays the same order. In this case, we sorted unit\_prices and spices in decreasing order. Next we go through each knapsack in line 90 using the same parsing method from lines 21-84 since I was unable to deep clone the array of Spice objects and there associated attributes.  On line 159,  I then applied a greedy algorithm. The complexity is $O(nlogn)$ since it takes $logn$ for each item and there is a total of $n$ items. Additionally, the $nlogn$ derives from the greedy choice part of the algorithm which is in the while loop.
    
\newpage
    
\section{RelativeInsertionSort Class}
    \begin{lstlisting}
public class RelativeInsertionSort_Decreasing {
    public void relative_insertionSort_decreasing(float[] A, Spice[] B){
        int n = A.length;
        for(int i = 1; i < n; i++){
            float key = A[i];
            Spice keyB = B[i];
            int j = i-1;
//            comparisons += 1;
            // if A[j] does not need to be switched, skip while
            while(j >= 0 && A[j] < key){
                A[j+1] = A[j];
                B[j+1] = B[j];
                j = j-1;
            }
            A[j+1] = key;
            B[j+1] = keyB;
        }
    }
}

    \end{lstlisting}
    Used in SSSP Class
    
    \section{Spice Class}
    \begin{lstlisting}
public class Spice {
    String color = null;
    float total_price = -1;
    int quantity = -1;
    float unit_price = -1;
}
    \end{lstlisting}
    Used in Heist Class
    
\section{Testing Class}
\begin{lstlisting}

    import java.io.*;
import java.util.Arrays;

public class Test {



    public static void main(String[] args) {
        String[] lines = {};
        // Read line by line the txt file using File reader
        String fileName = "graphs2";  //REMEMBER TO NOT HARDCODE
        File file = new File(fileName);
        try {
            FileReader fr = new FileReader(file);
            BufferedReader br = new BufferedReader(fr);
            String line;
            while ((line = br.readLine()) != null) {
                // add strings from txt file line by line into array words
                lines = Arrays.copyOf(lines, lines.length + 1); //extends memory
                lines[lines.length - 1] = line; //adds line to extra memory
            }
        } catch (FileNotFoundException e) {
            System.out.println("An error occurred.");
            e.printStackTrace();
        } catch (IOException e) {
            e.printStackTrace();
        }

        String[] copy_graphs0 = lines.clone();

        // 1)
        LinkedObjects out = new LinkedObjects();
        // outputs matrices
        System.out.println("GRAPH AS LINKED OBJECTS TESTING: ");
        out.linked_objects(copy_graphs0);
        int j = 0;
        for (int[][] i: out.matrices(copy_graphs0)){
            // i.length = |E|, out.num_vertices.get(j) = |V|, i = matrix, 1 = node source
            SSSP.bellman_ford(i, out.num_vertices.get(j), i.length, 1);
            j+=1;
        }

        // 2) Knapsacks
        String[] lines2 = {};
        // Read line by line the txt file using File reader
        String fileName2 = "spice";  //REMEMBER TO NOT HARDCODE
        File file2 = new File(fileName2);
        try {
            FileReader fr = new FileReader(file2);
            BufferedReader br = new BufferedReader(fr);
            String line;
            while ((line = br.readLine()) != null) {
                // add strings from txt file line by line into array words
                lines2 = Arrays.copyOf(lines2, lines2.length + 1); //extends memory
                lines2[lines2.length - 1] = line; //adds line to extra memory
            }
        } catch (FileNotFoundException e) {
            System.out.println("An error occurred.");
            e.printStackTrace();
        } catch (IOException e) {
            e.printStackTrace();
        }

        String[] spice = lines2.clone();
        Heist.spice_heist(spice);
    }
}
\end{lstlisting}
Where I read the .txt files and did the testing.
\end{document}


